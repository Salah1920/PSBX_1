\documentclass[]{article}
\usepackage{lmodern}
\usepackage{amssymb,amsmath}
\usepackage{ifxetex,ifluatex}
\usepackage{fixltx2e} % provides \textsubscript
\ifnum 0\ifxetex 1\fi\ifluatex 1\fi=0 % if pdftex
  \usepackage[T1]{fontenc}
  \usepackage[utf8]{inputenc}
\else % if luatex or xelatex
  \ifxetex
    \usepackage{mathspec}
  \else
    \usepackage{fontspec}
  \fi
  \defaultfontfeatures{Ligatures=TeX,Scale=MatchLowercase}
\fi
% use upquote if available, for straight quotes in verbatim environments
\IfFileExists{upquote.sty}{\usepackage{upquote}}{}
% use microtype if available
\IfFileExists{microtype.sty}{%
\usepackage[]{microtype}
\UseMicrotypeSet[protrusion]{basicmath} % disable protrusion for tt fonts
}{}
\PassOptionsToPackage{hyphens}{url} % url is loaded by hyperref
\usepackage[unicode=true]{hyperref}
\hypersetup{
            pdftitle={Convolutional Neural Network (CNN)},
            pdfborder={0 0 0},
            breaklinks=true}
\urlstyle{same}  % don't use monospace font for urls
\usepackage[margin=1in]{geometry}
\usepackage{graphicx,grffile}
\makeatletter
\def\maxwidth{\ifdim\Gin@nat@width>\linewidth\linewidth\else\Gin@nat@width\fi}
\def\maxheight{\ifdim\Gin@nat@height>\textheight\textheight\else\Gin@nat@height\fi}
\makeatother
% Scale images if necessary, so that they will not overflow the page
% margins by default, and it is still possible to overwrite the defaults
% using explicit options in \includegraphics[width, height, ...]{}
\setkeys{Gin}{width=\maxwidth,height=\maxheight,keepaspectratio}
\IfFileExists{parskip.sty}{%
\usepackage{parskip}
}{% else
\setlength{\parindent}{0pt}
\setlength{\parskip}{6pt plus 2pt minus 1pt}
}
\setlength{\emergencystretch}{3em}  % prevent overfull lines
\providecommand{\tightlist}{%
  \setlength{\itemsep}{0pt}\setlength{\parskip}{0pt}}
\setcounter{secnumdepth}{0}
% Redefines (sub)paragraphs to behave more like sections
\ifx\paragraph\undefined\else
\let\oldparagraph\paragraph
\renewcommand{\paragraph}[1]{\oldparagraph{#1}\mbox{}}
\fi
\ifx\subparagraph\undefined\else
\let\oldsubparagraph\subparagraph
\renewcommand{\subparagraph}[1]{\oldsubparagraph{#1}\mbox{}}
\fi

% set default figure placement to htbp
\makeatletter
\def\fps@figure{htbp}
\makeatother


\title{Convolutional Neural Network (CNN)}
\author{}
\date{\vspace{-2.5em}}

\begin{document}
\maketitle

\subsection{1 Introduction}\label{introduction}

Les reseaux de neurones convolutifs designent une sous-categorie de
reseaux de neurones et sont a ce jour un des modeles de classification
d'images reputes etre les plus performan au monde. Leur mode de
fonctionnement est a premiere vue simple : l'utilisateur fournit en
entree une image sous la forme d'une matrice de pixels. Elle a 2
dimensions pour une image en niveaux de gris. La couleur est representee
par une troisieme dimension, de profondeur 3 pour representer les
couleurs fondamentales {[}Rouge, Vert, Bleu{]}.

La disponibilite d'une grande quantite de donnees et l'amelioration de
la technologie materielle accelere la recherche sur les CNN, et des
architectures CNN approfondies recemment interessantes ont ete signale.
Plusieurs idees inspirantes pour faire avancer les CNN ont ete
explorees, telles que utilisation de differentes fonctions d'activation
et de perte, optimisation des parametres, regularisation et innovations
architecturales.

\subsection{2 Synthese de l'article}\label{synthese-de-larticle}

Cet article passe en revue les progres des architectures CNN, notamment
basees sur les modeles de conception des unites de traitement et a
propose la taxonomie des architectures CNN recentes. En plus de la
categorisation des CNN en differentes classes, cet article couvre
egalement l'histoire des CNN, ses applications, ses defis et directions
futures. Montre aussi que la capacite d'apprentissage de CNN s'est
considerablement amelioree au fil des ans en exploitant profondeur et
autres modifications structurelles.

\subsection{3 L'architecture d'un CNN comporte 2
parties}\label{larchitecture-dun-cnn-comporte-2-parties}

\subsection{3.1 Une partie convolutive}\label{une-partie-convolutive}

L'objectif final de cette partie est d'extraire des caracteristiques
propres a chaque image en les compressant de facon a reduire leur taille
initiale. En resume, l'image fournie en entree passe a travers une
succession de filtres, creant par la meme occasion de nouvelles images
appelees cartes de convolutions. Enfin, les cartes de convolutions
obtenues sont concatenees dans un vecteur de caracteristiques appele
code CNN.

\subsection{3.2 Une partie
classification}\label{une-partie-classification}

Le code CNN obtenu en sortie de la partie convolutive est fourni en
entree dans une deuxieme partie, constituee de couches entierement
connectees appelees perceptron multicouche (MLP pour Multi Layers
Perceptron). Le role de cette partie est de combiner les
caracteristiques du code CNN afin de classer l'image.

\subsection{4 Methode de sous echantillonnage : le
Max-Pooling}\label{methode-de-sous-echantillonnage-le-max-pooling}

Le Max-Pooling est un processus de discretisation base sur des
echantillons. Son objectif est de sous-echantillonner une representation
d'entree (image, matrice de sortie de couche cachee, etc.) en reduisant
sa dimension. De plus, son interet est qu'il reduit le cout de calcul en
reduisant le nombre de parametres a apprendre et fournit une invariance
par petites translations.

\subsection{5 Les differentes couches d'un
CNN}\label{les-differentes-couches-dun-cnn}

Avant de presenter les differentes couches d'un CNN on va expliquer
c'est quoi une Feature.

\subsection{5.1 Une Feature}\label{une-feature}

Une feature est vue comme un filtre : les deux termes sont equivalents
dans ce contexte. C'est ou on trouve toute la force des reseaux de
neurones convolutifs : ceux-ci sont capables de determiner tout seul les
elements discriminants d'une image, en s'adaptant au probleme pose. Par
exemple, si la question est de distinguer les chats des chiens, les
features automatiquement definies peuvent decrire la forme des oreilles
ou des pattes.

\subsection{5.2 Couche de convolution
(CONV)}\label{couche-de-convolution-conv}

Le role de cette premiere couche est d'analyser les images fournies en
entree et de detecter la presence d'un ensemble de features. On obtient
en sortie de cette couche un ensemble de features maps. La couche de
convolution est la composante cle des reseaux de neurones convolutifs,
et constitue toujours au moins leur premiere couche. Son but est de
reperer la presence d'un ensemble de features dans les images recues en
entree. Pour cela, on realise un filtrage par convolution: le principe
est de faire ``glisser'' une fenetre representant la feature sur
l'image, et de calculer le produit de convolution entre la feature et
chaque portion de l'image balayee.

\subsection{5.3 Couche de Pooling (POOL)
:}\label{couche-de-pooling-pool}

La couche de Pooling est une operation generalement appliquee entre deux
couches de convolution. Celle-ci recoit en entree les features maps
formees en sortie de la couche de convolution et son role est de reduire
la taille des images, tout en preservant leurs caracteristiques les plus
essentielles. Parmi les plus utilises, on retrouve le max-pooling
mentionne precedemment ou encore l'average pooling dont l'operation
consiste a conserver a chaque pas, la valeur moyenne de la fenetre de
filtre.

\subsection{5.4 La couche d'activation ReLU (Rectified Linear Units)
:}\label{la-couche-dactivation-relu-rectified-linear-units}

Cette couche remplace toutes les valeurs negatives recues en entrees par
des zeros. L'interet de ces couches d'activation est de rendre le modele
non lineaire et de ce fait plus complexe.

\subsection{5.5 Couche Fully Connected (FC)
:}\label{couche-fully-connected-fc}

Ces couches sont placees en fin d'architecture de CNN et sont
entierement connectees a tous les neurones de sorties (d'ou le terme
fully-connected). Apres avoir recu un vecteur en entree, la couche FC
applique successivement une combinaison lineaire puis une fonction
d'activation dans le but final de classifier l'input image (voir schema
suivant). Elle renvoie enfin en sortie un vecteur de taille d
correspondant au nombre de classes dans lequel chaque composante
represente la probabilite pour l'input image d'appartenir a une classe.

\subsection{6 Le template matching avec les filtres (formule
mathematique)}\label{le-template-matching-avec-les-filtres-formule-mathematique}

Les filtres sont egalement souvent utilises pour retrouver des motifs
particuliers dans une image. Ces motifs sont representes par de petites
images, appelees templates. La tache de template matching a pour but de
retrouver des templates dans une image. Le template matching realise
avec des filtres utilise l'operateur de correlation croisee
(cross-correlation). Cet operateur transforme l'image de representation
matricielle X en une nouvelle image Y de la facon suivante :

Dans ce contexte, H est une petite image representant le template a
retrouver. Concretement, cette operation revient a faire glisser H sur
l'image X, a multiplier les pixels qui se superposent et a sommer ces
produits. Ainsi, le template matching consiste a calculer la correlation
croisee entre une image X et un filtre dont le noyau H represente un
template que l'on souhaite retrouver dans X.

\subsection{7 Conclussion}\label{conclussion}

CNN a fait des progres remarquables, en particulier dans le traitement
d'image et les taches liees a la vision, et a ainsi ravive l'interet des
chercheurs pour les RNA. Dans ce contexte, plusieurs travaux de
recherche ont a ete menee pour ameliorer les performances de CNN sur ces
taches. Les progrrs des CNN peuvent etre categorise de differentes
manieres, y compris l'activation, la fonction de perte, l'optimisation,
la regularisation, algorithmes d'apprentissage et innovations en
architecture.

\subsection{8 Source}\label{source}

\url{https://blog.octo.com/classification-dimages-les-reseaux-de-neurones-convolutifs-en-toute-simplicite/}

\url{https://openclassrooms.com/fr/courses/4470531-classez-et-segmentez-des-donnees-visuelles/5082166-quest-ce-quun-reseau-de-neurones-convolutif-ou-cnn}

\url{https://datascientest.com/convolutional-neural-network}

\url{https://openclassrooms.com/fr/courses/4470531-classez-et-segmentez-des-donnees-visuelles/5083336-decouvrez-les-differentes-couches-dun-cnn}

\url{https://blog.octo.com/classification-dimages-les-reseaux-de-neurones-convolutifs-en-toute-simplicite/}

\url{https://openclassrooms.com/fr/courses/4470531-classez-et-segmentez-des-donnees-visuelles/5044811-decouvrez-la-notion-de-features-dans-une-image}

\url{https://fr.wikipedia.org/wiki/R\%C3\%A9seau_neuronal_convolutif}

\url{https://medium.com/@CharlesCrouspeyre/comment-les-r\%C3\%A9seaux-de-neurones-\%C3\%A0-convolution-fonctionnent-b288519dbcf8}

\end{document}
